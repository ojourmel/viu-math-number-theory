% Oliver Jourmel
%
%
% MATH 362 Number Theory
% Fall 2015
%
\documentclass[fleqn,letterpaper,10pt,twoside]{report}

%------------------------------------------------------------------------------
% DOCUMENT MARGINS, SPACING, AND FONTS
%------------------------------------------------------------------------------
%\usepackage[margin=2.6cm]{geometry}
%\usepackage{float}
\usepackage[normalem]{ulem}
%\usepackage{todonotes}
%\usepackage{soul}
%\linespread{2.1}

% === TODO old spec === %
\topmargin=-2cm
\textheight=24cm
\textwidth=16cm
\oddsidemargin=0cm
\evensidemargin=0cm

\setlength{\parindent}{0mm}

% Packages
\usepackage{booktabs}
\usepackage{pifont}
\usepackage{graphicx}
\usepackage{psfrag}
\usepackage{color}
\usepackage{colortbl}
\usepackage{fancyhdr}
\usepackage{parskip}
\usepackage{latexsym}
\usepackage{enumerate}
\usepackage{amssymb}
\usepackage{amsmath}
\usepackage{amsthm}
\usepackage{bm}
\usepackage{cancel}
\usepackage{appendix}
\usepackage{array}
\usepackage[bf]{titlesec}
\usepackage{tabto}

\usepackage{todonotes}

% Hyper links, with black links
\usepackage{hyperref}
\hypersetup{
    colorlinks=true,        % false: boxed links; true: coloured links
    linkcolor=black,        % colour of internal links
                            % (change box colour with linkbordercolor)
    citecolor=black,        % colour of links to bibliography
    filecolor=black,        % colour of file links
    urlcolor=black          % colour of external links
}

% Heading Formats
\titleformat{\chapter}[display]
{\sffamily\huge\bfseries}{Chapter \thechapter}{1ex}{}

\titleformat{\section}
{\sffamily\large\bfseries}{\thesection}{1em}{}

\titleformat{\subsection}
{\sffamily\normalsize\bfseries}{\thesubsection}{1em}{}

\titleformat{\subsubsection}
{\sffamily\normalsize\bfseries}{\thesubsubsection}{1em}{}


%------------------------------------------------------------------------------
% CUSTOM ENVIRONMENTS, HEADINGS, AND SYMBOLS
%------------------------------------------------------------------------------
\newcounter{example}
\newenvironment{example}
{\underline{\sffamily\bfseries Example \theexample}\\}
{\hspace*{\fill}$\bullet$ \stepcounter{example}}

\theoremstyle{plain}
\newtheorem{theorem}{Theorem}[subsection]
\newtheorem{corollary}{Corollary}[theorem]
\newtheorem{lemma}[theorem]{Lemma}

\theoremstyle{definition}
\newtheorem{definition}{Definition}[section]

\theoremstyle{remark}
\newtheorem*{fact}{Fact}
\newtheorem*{remark}{Remark}

\newcommand{\len}{\mathrm{len}}


%------------------------------------------------------------------------------
% BIBLIOGRAPHY SETUP
%------------------------------------------------------------------------------
\usepackage[english]{babel}
\usepackage{csquotes}
\usepackage{biblatex}
\renewcommand*{\nameyeardelim}{\addcomma\space}
\renewbibmacro{in:}{}
\addbibresource{vpn-sic-2015.bib}

%------------------------------------------------------------------------------
% APPENDIX SETUP
%------------------------------------------------------------------------------
%\usepackage[toc]{appendix}


%%%%%%%%%%%%%%%%%%%%%%%%%%%%%%%%%%%%%%%%%%%%%%%%%%%%%%%%%%%%%%%%%%%%%%%%%%%%%%

\begin{document}

\pagestyle{fancy}

\renewcommand{\chaptermark}[1]{\markboth{\chaptername \ \thechapter.\ #1}{}} 
\renewcommand{\sectionmark}[1]{\markright{\thesection.\ #1}{}}
\fancyhead[LE,RO]{\sffamily\bfseries \rightmark}
\fancyhead[LO,RE]{\sffamily\bfseries \leftmark}

\setcounter{chapter}{0}
\setcounter{definition}{1}
\setcounter{example}{1}

\chapter{Introduction}
\section{Number Theory}

\begin{definition}{Number Theory}

   \textit{Number Theory} is the theory of integers.
\end{definition}

\subsection{Roots of Number Theory}
Roots of number theory lie in:
\begin{itemize}
   \item{Euclid's Elements (300 BC)}
      \begin{itemize}
         \item{Prime numbers}
         \item{Composite numbers}
         \item{Greatest Common Deviser}
         \item{Properties of divisibility}
         \item{Perfect numbers}
      \end{itemize}
   \item{Diophantus (200AD - 300AD) (Father of Algebra)}
      \begin{itemize}
         \item{Diophantine Equations}
            Equations with integer solutions
            \begin{enumerate}
               \item{\(3x + 18y = 71\)}
               \item{\(x^2 + y^2 = z^2\)}
               \item{\(x^n + y^n = z^n, (n >= 3)\)}
            \end{enumerate}
      \end{itemize}
   \item{\href{https://en.wikipedia.org/wiki/Pierre\_de\_Fermat}{Fermat} (1601 - 1665)}
   \item{\href{https://en.wikipedia.org/wiki/Leonhard\_Euler}{Euler} (1707 - 1783)}
   \item{\href{https://en.wikipedia.org/wiki/Carl\_Friedrich\_Gauss}{Gauss} (1777 - 1855)}
   \item{\href{https://en.wikipedia.org/wiki/Joseph-Louis\_Lagrange}{Lagrange} (1736 - 1813)}
\end{itemize}

\subsubsection{Perfect Numbers}

\begin{center}
\begin{tabular}{l|llll}
   \#     &  \multicolumn{3}{c}{Sum of the Proper Divisors} & \\ \hline 
   2     &  \(1\) &  =  &  1 & \\
   3     &  \(1\) &  =  &  1 & \\
   4     &  \(1 + 2\) &  =  &  3 & \\
   5     &  \(1\) &  =  &  1 & \\
   6     &  \(1 + 2 + 3\) &  =  &  6 & * \\
   7     &  \(1\) &  =  &  1 & \\
   8     &  \(1 + 2 + 4\) &  =  &  7 & \\
   12    &  \(1 + 2 + 3 + 4 + 6\) &  =  &  16 & \\
   28    &  \(1 + 2 + 4 + 7 + 14\) &  =  &  28 & * \\
\end{tabular}
\end{center}

\(2,3,4,5,7,8\) are \textit{defitient}

\(12\) is \textit{abundant}

\(6,28\) are \textit{perfect}


\subsubsection{Lagrange - Interesting Problem}
\begin{center}
\begin{tabular}{rcl}
  1 & = & \(1^2 + 0^2\) \\
  1 & = & \(1^2 + 1^2\) \\
   \sout{3} & & \\
  1 & = & \(2^2 + 0^2\) \\
  1 & = & \(2^2 + 1^2\) \\
   \sout{6} & & \\
   \sout{7} & & \\
  8 & = & \(2^2 + 2^2\) \\
  8 & = & \(3^2 + 0^2\) \\
  10 & = & \(3^2 + 1^2\) \\
   \sout{11} & & \\
\end{tabular}
\end{center}

\textit{Which number can be or cannot be written as a sum of two squares?}

\subsubsection{An Additional Interesting Problem}
\begin{center}
\begin{tabular}{cccccccccc}
   1 & 4 & 8 & 9 & 16 & 25 & 27 & 32 & 36 & 49 \\
   \(1^2\) & \(2^2\) &\(2^3\) & \(3^2\) & \(4^2\) & \(5^2\) &\(3^3\) & \(2^5\) & \(6^2\) & \(7^2\) \\
\end{tabular}
\end{center}

\textit{Which number can be or cannot be written as a sum of two squares?} (like 8,9)

or

\textit{Solve this dophantine Equation:}
\[x^n-y^m=1\]

\subsection{Basics (Prime Numbers and Divisibility)}
\subsubsection{Important Unsolved Problem}
\textit{Is there a \emph{good} formula for determining the next prime, or even another prime?}

\textbf{Possible Formulas (Not that Good)}
\begin{itemize}
   \item{Sierpinski} \newline
      Let \(\xi = \sum\limits_{n=1}^\infty P_n10^{-2^n} \;, P_n = n^{\text{th}}\) prime

      \(\xi = 0.020300050000000700\ldots\)

      Then

      \(P_n = \left \lfloor{10^{2^n} \xi} \right \rfloor - 10^{2^{n_-1}} \cdot
      \left \lfloor{10^{2^{n_-1}} \xi} \right \rfloor \)

   \item{John Conway}
      \begin{center}
      \begin{tabular}{cccccccccccccc}
         A&B&C&D&E&F&G&H&I&J&K&L&M&N \\
         \(\frac{17}{91}\) &
         \(\frac{78}{85}\) &
         \(\frac{19}{51}\) &
         \(\frac{23}{38}\) &
         \(\frac{29}{33}\) &
         \(\frac{77}{29}\) &
         \(\frac{95}{23}\) &
         \(\frac{77}{19}\) &
         \(\frac{1 }{7 }\) &
         \(\frac{11}{13}\) &
         \(\frac{13}{11}\) &
         \(\frac{15}{2 }\) &
         \(\frac{1 }{7 }\) &
         \(\frac{55}{1 }\) \\
      \end{tabular}
      \end{center}

      Start with 2

      Multiply each fraction by 2 until you find the left most fraction where the product is an integer.

      \(2 \times \frac{15}{2} = 15\)

      Repeat this process using 15

      \(15 \times \frac{55}{1} = 825\)

      Continue with this definitely

      This produces:

      \(2,15,825,725,1925,2275,425,390,330,290,770,910,170,156,132,116,308,364,68,\boldsymbol{4},30,225,\ldots,136,\boldsymbol{8},\ldots,\boldsymbol{32},\ldots\)

      Notice that \(4 - 2^2, 8 = 2^3, 32 = 2^5\)

      If you list the powers of two, in the order that you find them, you get:

      \(2^2, 2^3, 2^5, 2^7, 2^11, 2^13, 2^17, 2^19, \ldots\)

\end{itemize}

\subsubsection{Divisibility}
\begin{definition}

   Let \(a\) and \(b\) be integers. We say that \(a\) \textit{devides}
   \(b\), written \(a \mid b\), if there is an integer \(c\) such that
   \(a \cdot c = b\)

   We might also say
   \begin{itemize}
      \item{\(a\) is a \textit{divisor} of \(b\)}
      \item{\(a\) is a \textit{factor} of \(b\)}
      \item{\(b\) is \textit{divisible} by \(a\)}
      \item{\(b\) is a \textit{multiple} of \(a\)}
   \end{itemize}

   If \(a\) does not divide \(b\), we write \(a \nmid b\).
\end{definition}

\todo{fix the formating when it comes to equations}
\begin{example}

   \(5 \mid 30 \),\enskip since \(5 \cdot 6 = 30 \)
   \(4 \nmid 60 \),\enskip since if \(4 \cdot c = 6 \), then \(c = \frac{3}{2}\) which is not an integer.

   \(4 \mid -4 \),\enskip since \(4 \cdot -1 = 4 \)
   \(-4 \mid 4 \),\enskip since \(-4 \cdot -1 = 4 \)

   \(0 \nmid 8 \),\enskip since if \(0 \cdot c = 8 \), then \(0 = 8\) which is a contradiction.

   \(8 \mid 0 \),\enskip since \(8 \cdot 0 = 0 \)

   \(0 \mid 30 \),\enskip since \(0 \cdot 0 = 0 \)

\end{example}

\begin{lemma}{\ }

   Let \(a, b\) and \(c\) be integers.

   If \(a \mid b\) and \(a \mid c\) \\
   Then
   \begin{enumerate}
      \item{\(a \mid b + c\)}
      \item{\(a \mid b - c\)}
      \item{\(a \mid b \cdot d\) for any integer \(d\)}
   \end{enumerate}

\end{lemma}

\begin{proof}{\ }

   \begin{enumerate}

      \item{}

         Suppose \(a \mid b\) and \(a \mid c\) \\
         Then there exists integers \(d\) and \(e\) such that
         \begin{equation*}
            \begin{split}
               a \cdot d & = b \\
               a \cdot e & = c \\
               a \cdot d + a \cdot e & = b + c \\
               a (d + e) & = b + c \\
            \end{split}
         \end{equation*}

         Since \((d + e)\) is an integer, as \(d\) and \(e\) are integers,
         \[a \mid b + c\]

      \item{\textbf{HomeWork}}

      \item{}

         Suppose \(a \mid b\) \\
         Then there exists an integers \(c\).\\
         Let \(d\) be some integer.
         \begin{equation*}
            \begin{split}
               a \cdot c & = b \\
               (a \cdot c) \cdot d & = b \cdot d \\
               a \cdot (c \cdot d) & = b \cdot d \\
            \end{split}
         \end{equation*}

         Since \((c \cdot d)\) is an integer,
         \[a \mid b \cdot d\]

   \end{enumerate}
\end{proof}


\end{document}


